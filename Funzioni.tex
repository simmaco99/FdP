\documentclass[a4paper,12pt,openany]{book}
\usepackage{hyperref}
\hypersetup{colorlinks=true,linkcolor=blue}

\usepackage[italian]{babel}
\usepackage{listings}
\lstloadlanguages{C}
\title{Funzioni utili in C sulle liste e sugli alberi }
\author{Simmaco Di Lillo \\ \href{mailto:dsimmaco@.com}{dsimmaco@gmail.com} 
 \and Giuseppe De Pasquale }

\begin{document}
\maketitle
Per la parte riguardanti le prove pratiche degli anni passati, ringrazio Giuseppe  De  Pasquale.\\
\tableofcontents

\chapter{Da copiare in ogni codice}
\begin{lstlisting}

#include<stdio.h>
#include<stdlib.h>

struct El{
  int info;
  struct El *next;
};

typedef struct El ElementoDiLista;
typedef ElementoDiLista* ListaDiElementi;
\end{lstlisting}

\chapter{Aggiungere}

\section{Aggiungere in testa}
\begin{lstlisting}

void AggiungiTesta(ListaDiElementi *lis, int val ){
	ListaDiElementi aux;
	 aux=malloc(sizeof(ElementoDiLista));
 	aux->info=val;
 	aux->next=*lis;
 	*lis=aux;
}
\end{lstlisting}

\section{Aggiungere in coda}
 \begin{lstlisting}
void AggiungiCoda(ListaDiElementi *lis, int val){
  if (*lis==NULL)
    {
      ListaDiElementi aux;
      aux=malloc(sizeof(ElementoDiLista));
      aux->info=val;
      aux->next=NULL;
      *lis=aux;
    }
  else
    AggiungiCoda(&((*lis)->next),val);
}


\end{lstlisting}
\newpage

\section{Aggiungere in ordine crescente}
Serve la funzione Aggiungi Testa 
 \begin{lstlisting}
void swap(ListaDiElementi *lis, int val)
{
    if((*lis==NULL) || ((*lis)->info>val) )
    AggiungiTesta(lis, val);
  else
    swap(&((*lis)->next),val);
}

\end{lstlisting}
 \section{Aggiungere -1 quando si verifica una condizione}
 \begin{lstlisting}
void Modifica(ListaDiElementi *lis){
  if (*lis!=NULL){
    if (condizione){
      ListaDiElementi aux;
      aux=malloc(sizeof(ElementoDiLista));
      aux->info=-1;
      aux->next=*lis;
      *lis=aux;
      Modifica(&((*lis)->next->next));}
    else
      Modifica(&((*lis)->next));
  }}


\end{lstlisting}
\chapter{Eliminare}

\section{Eliminare la testa}
 \begin{lstlisting}
void EliminaTesta(ListaDiElementi *lis){
  if (*lis!=NULL)
    {
      ListaDiElementi el=*lis;
        *lis=(*lis)->next;
      free(el);
    }
}
\end{lstlisting}

\section{Eliminare la  coda}
 \begin{lstlisting}
void EliminaCoda(ListaDiElementi *lis){
  if (*lis!=NULL)
    {
      if ((*lis)->next==NULL)
	*lis=NULL;
      else
	EliminaCoda(&((*lis)->next));
    }
}

\end{lstlisting}
\newpage

\section{Eliminare  la prima occorrenza }
 \begin{lstlisting}
 void EliminaPrimaOcc(ListaDiElementi *lis, int val){
  if(*lis!=NULL)
    {
      if((*lis)->info==val)
	{
	  ListaDiElementi el=*lis;
	  *lis=(*lis)->next;
	  free(el);
	}
      else
	EliminaPrimaOcc(&((*lis)->next),val);
    }
}
\end{lstlisting}

\section{Eliminare tutte le occorrenze}
 \begin{lstlisting}
void EliminaOcc(ListaDiElementi *lis, int val)
{
  if(*lis!=NULL)
    {
      if ((*lis)->info==val)
	{
	  ListaDiElementi el=*lis;
	  *lis=(*lis)->next;
	  free(el);
	  EliminaOcc(lis,val);  //2 istruzioni prima siamo andati avanti
	}
      else
	EliminaOcc(&((*lis)->next),val);
    }
}

\end{lstlisting}
\newpage

\section{Eliminare l'ultima occorrenza}
Serve la funzione per eliminare la testa e la coda. 
 \begin{lstlisting}

void EliminaUltima(ListaDiElementi *lis, int val)
{
  if (*lis!=NULL)
    {
      ListaDiElementi prec=NULL, occ=*lis,ultimo;
// ultimo  salva il prec di quello da eliminare (se ultimo=NULL va eliminata la testa)
      int trovato=0;
      while( occ->next!=NULL)
	{
	  if (occ->info==val)
	    {
	      trovato=1;
	      ultimo=prec;
	    }
	  prec=occ;
	  occ=occ->next;
	}
      if(occ->info==val)
	EliminaCoda(lis);
      else
	{
	  if (trovato)
	    {
	      if(ultimo==NULL)
		EliminaTesta(lis);
	      else
		{
		  ListaDiElementi el=ultimo->next;
		  ultimo->next=(ultimo->next)->next;
		  free(el);
		}
	    }
	}
    }
}

\end{lstlisting}
\chapter{Altre funzioni}
\section{Stampare }
 \begin{lstlisting}
void PrintList(ListaDiElementi lis)
{
  if (lis==NULL)
    printf("NULL\n");
  else
    {
      printf("%d -> ", lis->info);
      PrintList(lis->next);
    }
}


\end{lstlisting}

\section{Costruire una lista invertita}
Serve la funzione per agginugere in testa
 \begin{lstlisting}
ListaDiElementi ListaInversa(ListaDiElementi lis){
  ListaDiElementi ris=NULL;
  if(lis!=NULL)
    {
      ListaDiElementi occ=lis;
      while(occ!=NULL)
	{
	AggiungiTesta(&ris,occ->info);
	occ=occ->next;
	} }
  return ris;}


\end{lstlisting}

\section{Verificare la presenza di un valore}
Da aggiungere questa linea di codice
 \begin{lstlisting}
  typedef enum{false, true} boolean;
  
 \end{lstlisting}
 \begin{lstlisting}

boolean Trova(ListaDiElementi lis, int val ){
    boolean ris=false;
    if(lis!=NULL)
    {
        if (lis->info==val)
            ris=true;
        else
            ris=Trova(lis->next,val);
        
    }
    return ris;
}
\end{lstlisting}

\section{ Contare le occorrenze di un valore}
 \begin{lstlisting}
int ContaOcc(ListaDiElementi lis,int val)
{
  int ris=0;
  if(lis!=NULL)
    {
      if (lis->info==val)
	ris=1+ContaOcc(lis->next, val);
      else
	ris=ContaOcc(lis->next,val);
    }
  return ris;
}
\end{lstlisting}
\newpage


\section{Contare il numero di elementi}
 \begin{lstlisting}
int Lunghezza(ListaDiElementi lis){
  int ris=0;
  if (lis!=NULL)
    ris=1+Lunghezza(lis->next);
  return ris;
}

\end{lstlisting}


\part{Alberi}
\chapter{Da copiare in ogni programma}

 \begin{lstlisting}
#include<stdio.h>
#include<stdlib.h>
struct nodoAlberoBinario{
  int label;
  struct nodoAlberoBinario *left;
  struct nodoAlberoBinario *right;
};

typedef struct nodoAlberoBinario NodoAlbero;

typedef NodoAlbero *AlberoBinario;
  \end{lstlisting}
\chapter{Visite}  
  \section{Simmetrica}
  La visita procede nel seguente ordine 
  \begin{enumerate}
  \item Ramo di sinistra
  \item Radice 
  \item Ramo di destra
  \end{enumerate}
  Se si visita un albero di ricerca in questo modo si ottiene una lista di elementi in ordine crescente
   \begin{lstlisting}
void VisitaSimmetrica(AlberoBinario bt){
  if (bt !=NULL){
    VisitaSimmetrica(bt->left);
    printf("%d->", bt->label);
    // Modifica( &(bt->label)); 
    // modifica ha come prototipo Modifica (int * x)
    VisitaSimmetrica(bt->right);
  }
}   
 \end{lstlisting}
\newpage
  \section{Anticipata}
   La visita procede nel seguente ordine 
  \begin{enumerate}
  \item Radice 
  \item Ramo di sinistra
  \item Ramo di destra
  \end{enumerate}
   \begin{lstlisting}
   void VisitaAnticipata(AlberoBinario bt){
  if (bt!=NULL)
    {
      printf("%d", bt->label);
      // qui si modifica qualcosa
      VisitaAnticipata(bt->left);
      VisitaAnticipata(bt->right);
    }}
 \end{lstlisting}

  \section{Posticipata}
   La visita procede nel seguente ordine 
  \begin{enumerate}
   \item Ramo di destra
    \item Ramo di sinistra
   \item Radice 
 \end{enumerate}
   \begin{lstlisting}
   void VisitaPosticipata(AlberoBinario bt){
  if (bt!=NULL){
    VisitaPosticipata(bt->right);
    VisitaPosticipata(bt->left);
    printf("%d", bt->label);
    // qui si modifica
  }}

 \end{lstlisting}
\chapter{Trovare elementi }
  \section{Controllare la presenza di un elemento}
  Da aggiungere questa linea di codice
 \begin{lstlisting}
  typedef enum{false, true} boolean;
  
 \end{lstlisting}
   \begin{lstlisting}
 boolean Controlla( AlberoBinario bt, int val){
    boolean ris=false;
    if(bt!=NULL)
        ris=( (bt->label== val) || Controlla( bt->left,val) || Controlla(bt->right, val));
    
    return ris;
}


 \end{lstlisting}
 \newpage
 \section{Trovare profondit\'a di un elemento}
  Nel caso che l'elemento non si trovi nell'albero la funzione restituisce $-1$
   \begin{lstlisting}
 int Profondita(AlberoBinario bt , int val)
{   int ris=-1;
    if(trova(bt,val))
    {   ris=0;
        if (bt->label==val)
            ris=0;
        else
        {
            if ((bt->label)>val)
                ris=1+prof(bt->left,val);
            else
                ris=1+prof(bt->right,val);
         }
    }
    return ris;
}  
 \end{lstlisting}

  \section{Contare le occorrenze di un elemento}
   \begin{lstlisting}
int ContaOcc(AlberoBinario bt, int val){
  int ris=0;
  if (bt !=NULL){
    if ((bt->label)==val)
      ris++;
    ris=ris+ContaOcc(bt->left,val)+ContaOcc(bt->right,val);
  }
  return ris;
}
	 
 \end{lstlisting}
\chapter{Altre funzioni}
  \section{Aggiungere elementi ad un albero di ricerca}
   \begin{lstlisting}
   void Aggiungi( AlberoBinario *bt, int val){
    if((*bt)==NULL){
        AlberoBinario aux;
        aux=malloc(sizeof(NodoAlbero));
        aux->label=val;
        aux->left=NULL;
        aux->right=NULL;
        *bt=aux;}
    else
    {
        if (((*bt)->label)>val)
            Aggiungi(&((*bt)->left),val);
        else
            Aggiungi(&((*bt)->right),val);
    }
}
 \end{lstlisting}

\newpage
  \section{Calcolare l'altezza di un albero }
  Per la funzione principale occorre la seguente funzione che restituisce il massimo tra 2 numeri
   \begin{lstlisting}
   int max (int a,int b)
{
    int ris=a;
    if(a<b)
        ris=b;
    return ris;
}
 \end{lstlisting}
 La funzione principale
 
   \begin{lstlisting}
int Altezza(AlberoBinario bt)
{
    int ris;
    if (bt==NULL)
        ris=0;
    else
        ris=max(Altezza(bt->left),Altezza(bt->right))+1;
    return ris;
}
 \end{lstlisting}
 
\part{Funzioni presenti in vecchie prove pratiche o esercitazioni}
\chapter{Funzioni Gennaio 2018}
  \section{ReadList}
    readList:  Legge dallo standard input una sequenza di numeri interi
ordinati in maniera strettamente crescente e termina automaticamente l’acquisizione alla prima occorrenza di un numero che non rispetta
l’ordinamento (l’intero che viola l’ordinamento non va inserito
nella lista).   Gli  interi  devono  essere  memorizzati,  nell’ordine  di  acquisizione, in una lista concatenata opportunamente allocata.  La funzione readList restituisce un intero pari al numero di elementi presenti nella lista creata.  Per esempio, supponendo che venga acquisita la sequenza (5,8,15,9) la lista dovrebbe essere la seguente (e sarebbe la stessa se si  acquisisse  la  sequenza  (5,8,15,15)),  e  la  funzione  restituirebbe  il valore 3:
	   \begin{lstlisting}
int readList(ListaDiElementi *lista)
{
	ListaDiElementi li=malloc(sizeof(ElementoLista));
	int a;
	int b;
	int c=1;
	scanf("%d", &a);
	AggiungiCoda(lista,a);
	b=a;
	scanf("%d", &a);
	while(a>b)
		{
		c=c+1;
		b=a;
		AggiungiCoda(lista,a);
		scanf("%d", &a);
		}
	return c;
}
	\end{lstlisting}
  \section{filterLists}
  filterList: date due liste list1 e list2 di lunghezza qualsiasi ordinate e che non contengono duplicati, la funzione elimina dalla lista list1 tutti gli elementi che sono presenti anche nella lista list2.
	   \begin{lstlisting}
void filterLists(ListaDiElementi *lista1, ListaDiElementi lista2)
{
	if((*lista1)!=NULL&&lista2!=NULL)
		{
		if(((*lista1)->info)>(lista2->info))
			{
		filterLists(&((*lista1)->next),lista2);
			}
		else
			{
			if(((*lista1)->info)>(lista2->info))
				{
			filterLists(lista1,(lista2->next));
				}
			else
				{
			ListaDiElementi aux=(*lista1);
			(*lista1)=((*lista1)->next);
			filterLists(lista1,(lista2->next));
				}
			}
		}
}
		\end{lstlisting}

	\chapter{Funzioni Febbraio 2018}
  \section{ReadList}
 ReadList: 
 legge  una  sequenza  di  numeri  interi, 
  finch\ 'e  vale  la  condizione
  
    che  questa  alterni  numeri  pari  e  numeri  dispari  (cominciando indifferentemente per uno o l’altro),  e termina l’acquisizione quando legge il primo elemento che viola tale alternanza.  I numeri letti devono essere memorizzati in una lista nell’ordine inverso a quello di acquisizione (dunque con inserimento in testa):  l’ultimo numero letto, che fa terminare l’acquisizione, non va inserito nella lista. Per esempio, supponendo che la sequenza immessa sia (5,0,15,10,12),al termine di ReadList avremmo la lista (10,15,0,5).
Il  puntatore  alla  testa  della  lista  creata  deve  essere  restituito  come valore di ritorno della funzione.
   \begin{lstlisting}
ListaDiElementi readList()
{
	ListaDiElementi occ=malloc(sizeof(ElementoLista));
	int a;
	int b;
	scanf("%d", &a);
	(occ->info)=a;
	(occ->next)=NULL;
	b=a;
	scanf("%d", &a);
	while((b+a)%2==1)
		{
		b=a;
		ListaDiElementi lis=malloc(sizeof(ElementoLista));
		(lis->info)=a;
		(lis->next)=occ;
		occ=lis;
		scanf("%d", &a);
		}
	return occ;
} 
	\end{lstlisting}
  \section{OddEven}
  	OddEven:  Questa funzione prende una lista di interi e restituisce 1 se essa  contiene  lo  stesso  numero  di  elementi  pari  ed  elementi  dispari. Restituisce 0 altrimenti.
Per esempio, per la lista (10,15,0,5),OddEven restituir\'a 1.  Per la lista(1,8,3),OddEven restituir\'a 0.
	   \begin{lstlisting}
int OddEven(ListaDiElementi lista)
{
	int a=0;
	if(lista!=NULL)
		{
		a=1;
		while(lista->next!=NULL)
			{
			lista=lista->next;
			a=a+1;
			}
		}
	a=(a+1)%2;
	return a;
}
	\end{lstlisting}
  \section{Equality}
  	Equality:  data  una  lista  di  interi, Equality deve  cancellare
dalla coda  della  lista il  minimo numero  (che  pu`o  anche  essere  =  0)  di elementi affinch\'e la lista risultante contenga tanti numeri pari quanti numeri dispari. Per esempio,  se la lista fosse (1,8,3). Equality restituirebbe la lista (1,8). Il  puntatore  alla  testa  della  lista  creata  deve  essere  restituito  come valore di ritorno della funzione.
	   \begin{lstlisting}
void filterLists(ListaDiElementi *lista1, ListaDiElementi lista2)
{
	if((*lista1)!=NULL&&lista2!=NULL)
		{
		if(((*lista1)->info)>(lista2->info))
			{
			filterLists(&((*lista1)->next),lista2);
			}
		else
			{
			if(((*lista1)->info)>(lista2->info))
				{
				filterLists(lista1,(lista2->next));
				}
			else
				{
				ListaDiElementi aux=(*lista1);
				(*lista1)=((*lista1)->next);
				filterLists(lista1,(lista2->next));
				}
			}
		}
}
		\end{lstlisting}
		\chapter{Esercitazione 2018}
  \section{removemin}
  removeMin: data una lista list di lunghezza maggiore o uguale a uno e con possibili elementi duplicati, la funzione elimina dalla lista l’elemento minimo. Nel caso in cui ci siano elementi minimi ripetuti, la funzione elimina quello pi\'u a destra (ossia l’ultimo incontrato navigando la lista a partire da list). 
   \begin{lstlisting}
int removeMin(ListaDiElementi* lista)
{
	int a;
	int b;
	ListaDiElementi prec=NULL;
	ListaDiElementi occ=*lista;
	ListaDiElementi min=NULL;
	if(occ!=NULL)
		{
		a=((*lista)->info);
		if((*lista)->next!=NULL)
			{
			prec=occ;
			occ=occ->next;
			}
		while((occ->next)!=NULL)
			{
			b=occ->info;
			if(b<=a)
				{
				a=b;
				min=prec;
				}
			prec=occ;
			occ=occ->next;
			}
		b=occ->info;
		if(b<=a)
			{
			while(((*lista)->next)!=NULL)
				{
				lista=(&(*lista)->next);
				}
			(*lista)=NULL;
			a=b;
			}
		else
			{
			if(min==NULL)
				{
				ListaDiElementi el=*lista;
				*lista=((*lista)->next);
				free(el);
				}
			else
				{
				ListaDiElementi el=(min->next);
				(min->next)=(min->next)->next;
				free(el);
				}
			}
		return a;
		}
}
	\end{lstlisting}
	 \section{pushback}
pushBack: data una lista list di lunghezza maggiore o uguale a zero e un intero x, inserisce x in fondo alla lista.
   \begin{lstlisting}
void pushBack(ListaDiElementi* lista, int x)
{
	if(*lista==NULL)
		{
		ListaDiElementi aux;
		aux=malloc(sizeof(ElementoLista));
		aux->info=x;
		aux->next=NULL;
		*lista=aux;
		}
	else
		{
		pushBack(&((*lista)->next),x);
		}
}
	\end{lstlisting}
	
	\chapter{Gennaio 2019}
  \section{read}
read: La funzione read legge dallo standard input una sequenza di numeri interi ordinati in maniera strettamente crescente e termina automaticamente l’acquisizione alla prima occorrenza di un numero che non rispetta l’ordinamento (l’intero che viola l’ordinamento non va inserito nella lista).  Gli interi devono essere memorizzati, nell’ordine di acquisizione, in una lista concatenata opportunamente allocata (Nota:  Il puntatore l passato come argomento alla procedura— deve essere impostato in modo che punti alla lista creata).  La funzione read restituisce un intero pari al numero di elementi presenti nella lista creata.
   \begin{lstlisting}
int read(List *l)
{
	int a;
	int b;
	scanf("%d", &a);
	if((*l)!=NULL)
		{
		if(((*l)->info)<a)
			{
			List aux;
			aux=malloc(sizeof(Element));
			aux->info=a;
			aux->next=NULL;
			*l=aux;
			b=1+read(&aux);
			((*l)->next)=aux;
			return b;
			}
		else
			{
			*l=NULL;
			return 0;
			}
		}
	else
		{
		List aux;
		aux=malloc(sizeof(Element));
		aux->info=a;
		aux->next=NULL;
		*l=aux;
		b=1+read(&aux);
		((*l)->next)=aux;
		return b;
		}
}
	\end{lstlisting}
	  \section{intersection}
intersection: Date  due  liste l1 e l2 di  lunghezza  qualsiasi,  i  cui  elementi  sono in  ordine strettamente crescente e che non contengono duplicati, la procedura intersection deve modificare la lista l1 in modo che mantenga solo gli elementi comuni a entrambe le liste (ovvero eliminando tutti gli elementi della lista l1 che non sono presenti nella lista l2).
   \begin{lstlisting}
void  intersection(List *l1, List l2)
{
if((*l1)!=NULL&&l2!=NULL)
	{
	if((*l1)->next!=NULL&&(l2->next)!=NULL)
		{
		if((l2->info)<(*l1)->info)
			{
			intersection(l1,(l2->next));
			}
		else
	 		{
			if((l2->info)>(*l1)->info)
				{
				*l1=((*l1)->next);
				intersection(l1,l2);
				}
			else
				{
				intersection(&((*l1)->next),l2);
				}					
			}
		}
	else
		{
		if((l2->next)==NULL)
			{
			if((*l1)!=NULL&&((*l1)->info)==(l2->info))
				{
				((*l1)->next)=NULL;
				}
			else
				{
				if((*l1)!=NULL)
					{
					*l1=NULL;
					}
				}
			}
		}	
	}
}
	\end{lstlisting}
	\chapter{Gennaio 2019 primo appello}
	\section{insert}
	La procedura insert deve prendere in ingresso un albero binario di ricerca e un valore intero val. Se val non `e contenuto nell’albero, insert deve creare un nuovo nodo con val come valore, inserirlo nell’albero (mantenendo l’ordine dell’albero binario di ricerca). Altrimenti, insert non deve effettuare alcun inserimento nell’albero ricevuto in ingresso.
	\begin{lstlisting}
	void insert( AlberoBinario *radice, int val ){
  if((*radice)==NULL){
    AlberoBinario aux;
    aux=malloc(sizeof(NodoAlbero));
    aux->label=val;
    aux->left=NULL;
    aux->right=NULL;
    *radice=aux;}
  else
    {
      if (((*radice)->label)>val)
	insert(&((*radice)->left),val);
      if (((*radice)->label)<val)
        insert(&((*radice)->right),val);
    }
}


\end{lstlisting}

\section{kth}
La funzione kth deve prendere in input la radice di un albero binario di ricerca ed un valore k. La funzione kth deve restituire il k-esimo valore tra quelli memorizzati nell’albero di ricerca la cui radice `e radice. Si assuma che il numero n di interi memorizzati nell’albero sia non inferiore a due e che k ≤ n.\\
Attenzione serve la funzione ContaNodi
\begin{lstlisting}
int ContaNodi(AlberoBinario radice)
{
    int ris=0,dx,sx;
    if ( radice!=NULL )
    {
        return 1+ContaNodi(radice->left)+ ContaNodi(radice->right) ;
    }
    return ris;
}

int kth( AlberoBinario radice, int k){
    int ris;
    int n=ContaNodi(radice->left);
    if ( n==k-1)
       ris=radice->label;
    else
      {
          if ( n>k-1 )      ris=kth(radice->left,k);
          
          else              ris=kth(radice->right, k-(n+1));
          
      }
    return ris;
 
}
\end{lstlisting}
\end{document}